\documentclass {article}

\usepackage[a4paper]{geometry}
\usepackage[utf8]{inputenc}
\usepackage[T1]{fontenc}
\usepackage{amsmath,amssymb}

\newcommand {\mpc}{\texttt {mpc}}
\newcommand {\mpfr}{\texttt {mpfr}}
\newcommand {\ulp}[1]{#1~ulp}
\newcommand {\atantwo}{\operatorname {atan2}}
\DeclareMathOperator{\error}{error}
\newcommand {\Ulp}{{\rm ulp}}
\newcommand {\Exp}{{\rm \textsc exp}}

\title {MPC: Algorithms and Error Analysis}
\author {Andreas Enge \and Philippe Th\'eveny}
\date {April 11, 2008}

\begin {document}
\maketitle
\tableofcontents


\section {Error analysis}

This section is devoted to the analysis of error propagation: Given a function
whose input arguments already have a certain error, what is the error bound on
the function output? The output error usually consists of two components: the
error propagated from the input, which may be arbitrarily amplified; and an
additional small error accounting for the rounding of the output. The results
are needed for algorithms that combine several arithmetic operations.

We will assume that the real and imaginary parts of the operands have the same
precision $p$. We write any nonzero real number $x$ in the form $x = m \cdot
2^e$ with $\frac{1}{2} \le |m| < 1$ and $e := \Exp(x)$, and we define $\Ulp(x)
:= 2^{\Exp(x) - p}$.  We will also assume that the two operands $z_1$ and
$z_2$ in the following operations are approximations of exact values
$\widetilde{z_1}$ and $\widetilde{z_2}$ respectively. For $n=1, 2$, let
$\widetilde{z_n} = \widetilde{x_n} + i \widetilde{y_n}$ and $z_n = x_n + i
y_n$, let $k_{R,n}$ and $k_{I,n}$ the upper bound coefficients: $\error(x_n)
\leq k_{R,n} \Ulp(x_n)$ and $\error(y_n) \leq k_{I,n} \Ulp(y_n)$.  Note that
we always have $x_n.c_{R,n}^- \leq \widetilde{x_n} \leq x_n.c_{R,n}^+$ and
$y_n.c_{I,n}^- \leq \widetilde{y_n} \leq y_n.c_{I,n}^+$, with $c_{\cdot,n}^- =
1-k_{\cdot,n}2^{1-p}$ and $c_{\cdot,n}^+ = 1+k_{\cdot,n}2^{-p}$. When $z_n$ is
obtained by rounding $\widetilde{z_n}$ to the precision $p$ (we note
$z_n=\circ(\widetilde{z_n})$), we have another inequalities: $\frac{1}{2}|z_n|
\leq |\widetilde{z_n}| \leq 2|z_n|$; if the rounding mode is rounding towards
plus infinity $|\widetilde{z_n}| \leq |z_n|$, and if the rounding mode is
toward minus infinity $|z_n| \leq |\widetilde{z_n}|$.

Let $z=x+iy$ the result of the operation $z_1\diamond z_2$ rounded to the
precision of $z$ (we note $z=\circ(z_1 \diamond z_2)$). Let $c_R$ the bound
coefficient for the rounding error of the real part: $\error(x) = |x-\Re(z_1
\diamond z_2)| \leq c_R \Ulp(x)$, if the rounding mode is rounding to nearest
$c_R=\frac{1}{2}$, else $c_R=1$ (same relations for $c_I$, the coefficient for
the imaginary part, \emph{mutatis mutandis}).

\subsection {Generic error of addition/substraction}

Let
\[
z=\circ(z_1+z_2).
\]
We have
\begin{align*}
\error(x)&= |x-\Re(\widetilde{z_1}-\widetilde{z_2})|
\\
&\leq |x-\Re(z_1-z_2)|+|\Re(z_1+z_2)-\Re(\widetilde{z_1}+\widetilde{z_2})|
\\
&\leq c_R\Ulp(x)+|x_1-\widetilde{x_1}|+|x_2-\widetilde{x_2}|
\\
&\leq c_R\Ulp(x)+k_{R,1}\Ulp(x_1)+k_{R,2}\Ulp(x_2)
\\
&\leq \left[c_R+k_{R,1}2^{d_{R,1}}+k_{R,2}2^{d_{R,2}}\right]\Ulp(x)
\end{align*}
where $d_{R,n}=\Exp(x_n)-\Exp(x)$. If $x_1x_2 \geq 0$, then we have a simpler
expression
\[
\error(x) \leq [c_R+k_{R,1}+k_{R,2}] \Ulp(x).
\]
In the same way, we can show that the generic error of the imaginary part is
\[
\error (y) \leq \left[c_I+k_{I,1}2^{d_{I,1}}+k_{I,2}2^{d_{I,2}}\right] \Ulp(y)
\]
where $d_{I,n}=\Exp(y_n)-\Exp(y)$. If $y_1y_2 \geq 0$, then we have the
simpler expression
\[
\error(y) \leq [c_I+k_{I,1}+k_{I,2}] \Ulp(y).
\]


\subsection {Generic error of multiplication}

Let
\[
z=\circ(z_1\times z_2).
\]
We have
\[
\error(x) = |x-\Re(\widetilde{z_1}\times \widetilde{z_2})| \leq
|x-\Re(z_1\times z_2)|
+|\Re(z_1\times z_2)-\Re(\widetilde{z_1}\times\widetilde{z_2})|.
\]
The first term on the right hand side is the rounding error, so
\[
|x-\Re(z_1\times z_2)| \leq c_R \Ulp(x),
\]
the second term can be split as follows
\[
|\Re(z_1\times z_2)-\Re(\widetilde{z_1}\times\widetilde{z_2})| \leq
|x_1x_2-\widetilde{x_1}\widetilde{x_2}|
+|y_1y_2-\widetilde{y_1}\widetilde{y_2}|.
\]
Let us bound the part with real components:
\begin{align*}
|x_1x_2-\widetilde{x_1}\widetilde{x_2}| &\leq
\frac{1}{2}\left(|x_2||x_1-\widetilde{x_1}|
+|\widetilde{x_1}||x_2-\widetilde{x_2}|+|x_1||x_2-\widetilde{x_2}|
+|\widetilde{x_2}||x_1-\widetilde{x_1}|\right)
\\
&\leq \left[(1+c_{R,2})k_{R,1}+(1+c_{R,1})k_{R,2}\right] \Ulp(x_1x_2),
\end{align*}
where $c_{R,n}$ is such that $|\widetilde{z_n}| \leq c_{R,n}|z_n|$, it could
be $c_{R,n}^+=1+k_{R,n}2^{-p}$ or even 1 if $z_n$ is $\widetilde{z_n}$ rounded
towards plus infinity (see the introductory paragraphs above). In a similar
way, we have
\[
|y_1y_2-\widetilde{y_1}\widetilde{y_2}| \leq
\left[(1+c_{I,1})k_{I,2}+(1+c_{I,2})k_{I,1}\right] \Ulp(y_1y_2).
\]
If [$x_1x_2 \geq 0$ and $y_1y_2 \geq 0$] or [$x_1x_2 \leq 0$ and $y_1y_2 \leq
  0$] then $\Ulp(x_1x_2) \leq \Ulp(x)$ and $\Ulp(y_1y_2) \leq \Ulp(x)$, so we
have
\[
\error(x) \leq \left[c_R + (1+c_{R,1})k_{R,2} + (1+c_{R,2})k_{R,1} +
(1+c_{I,1})k_{I,2} + (1+c_{I,2})k_{I,1} \right] \Ulp(x).
\]
If $x_1x_2$ and $y_1y_2$ don't have the same sign, then the inequality is
\[
\error(x) \leq \left[ c_R + \left( (1+c_{R,1})k_{R,2} + (1+c_{R,2})k_{R,1}
\right)2^d + \left( (1+c_{I,1})k_{I,2} + (1+c_{I,2})k_{I,1} \right)2^{d'}
\right] \Ulp(x)
\]
where $d = \Exp(x_1x_2)-\Exp(x) \leq \Exp(x_1)+\Exp(x_2)-\Exp(x)$ and $d' =
\Exp(y_1y_2)-\Exp(x) \leq \Exp(y_1)+\Exp(y_2)-\Exp(x)$.

Error of the imaginary part can be bounded the same way. If $x_1y_2$ and
$x_2y_1$ {\bf do not have} the same sign, then
\[
\error(y) \leq \left[c_I + (1+c_{R,1})k_{I,2} + (1+c_{I,2})k_{R,1} +
(1+c_{I,1})k_{R,2} + (1+c_{R,2})k_{I,1}\right] \Ulp(y).
\]
If $x_1y_2$ and $y_1x_2$ {\bf have} the same sign, then
\[
\error(y) \leq \left[c_I + \left( (1+c_{R,1})k_{I,2} + (1+c_{I,2})k_{R,1}
\right)2^{\delta} + \left( (1+c_{I,1})k_{R,2} + (1+c_{R,2})k_{I,1} \right)
2^{\delta'}\right] \Ulp(y).
\]
where $\delta = \Exp(x_1y_2)-\Exp(y) \leq \Exp(x_1)+\Exp(y_2)-\Exp(y)$ and
$\delta' = \Exp(y_1x_2)-\Exp(y) \leq \Exp(y_1)+\Exp(x_2)-\Exp(y)$.

\subsection {Generic error of division}

Let
\[
z=\circ(\frac{z_1}{z_2}).
\]

We note
\begin{align*}
A&=\widetilde{x_1}\widetilde{x_2}+\widetilde{y_1}\widetilde{y_2}&
B&=\widetilde{y_1}\widetilde{x_2}-\widetilde{x_1}\widetilde{y_2}&
D&=\widetilde{x_2}^2+\widetilde{y_2}^2\\
a&=x_1x_2+y_1y_2&
b&=y_1x_2-x_1y_2&
d&=x_2^2+y_2^2
\end{align*}
then the error of the real part is
\[
\error(x) = |x-\frac{A}{D}| \leq |x-\frac{a}{d}| + |\frac{a}{d}-\frac{A}{D}|.
\]
The first term of the right hand side is the rounding error:
\[
|x-\frac{a}{d}|\leq c_R \Ulp(x).
\]
The second term can be bounded as follows:
\[
|\frac{a}{d}-\frac{A}{D}| \leq \frac{1}{d} |a-A|
+\left|\frac{A}{D}\right|\frac{1}{d}|d-D|.
\]
We have
\[
|a-A| \leq |x_1x_2-\widetilde{x_1}\widetilde{x_2}|
+|y_1y_2-\widetilde{y_1}\widetilde{y_2}|,
\]
with
\begin{align*}
|x_1x_2-\widetilde{x_1}\widetilde{x_2}| &\leq \frac{1}{2} \left(
|x_1||x_2-\widetilde{x_2}|+|\widetilde{x_2}||x_1-\widetilde{x_1}|
+|x_2||x_1-\widetilde{x_1}|+|\widetilde{x_1}||x_2-\widetilde{x_2}| \right)\\
&\leq \left[ (1+c_{R,1})k_{R,2}+(1+c_{R,2})k_{R,1} \right] \Ulp(x_1x_2),
\end{align*}
where $c_{r,n}$ is such that $|\widetilde{x_n}| \leq c_{R,n} |x_n|$ (see the
introductory paragraphs above) and using the folowing general rule
\[
|u|\cdot\Ulp(v) \leq 2\Ulp(uv);
\]
let $d=\Exp(x_1x_2)-\Exp(a)$, we have
\[
|x_1x_2-\widetilde{x_1}\widetilde{x_2}| \leq 
\left((1+c_{R,1})k_{R,2}+(1+c_{R,2})k_{R,1}\right)2^d \Ulp(a),
\]
and we can show in a similar way that
\[
|y_1y_2-\widetilde{y_1}\widetilde{y_2}| \leq \left( (1+c_{I,1})k_{I,2}
+(1+c_{I,2})k_{I,1} \right)2^{d'} \Ulp(a),
\]
where $d'=\Exp(y_1y_2)-\Exp(a)$. Using the preceding inequalities we have
\[
\frac{1}{d}|a-A| \leq \left[
\left((1+c_{R,1})k_{R,2}+(1+c_{R,2})k_{R,1}\right)2^{1+d}
+\left((1+c_{I,1})k_{I,2}+(1+c_{I,2})k_{I,1}\right)2^{1+d'}
\right] \Ulp(x).
\]
In addition, we have
\[
|d-D| \leq |x_2^2-\widetilde{x_2}^2| + |y_2-\widetilde{y_2}^2|,
\]
with
\begin{align*}
|x_2^2-\widetilde{x_2}| &\leq |x_2+\widetilde{x_2}||x_2-\widetilde{x_2}|\\
&\leq (1+c_{R,2})|x_2|k_{R,2} \Ulp(x_2)\\
&\leq 2(1+c_{R,2})k_{R,2}\Ulp(x_2^2),
\end{align*}
we can show in a similar way that
\[
|y_2^2-\widetilde{y_2}^2| \leq 2(1+c_{I,2})k_{I,2}\Ulp(y_2^2).
\]
But $\Ulp(x_2^2) \leq \Ulp(x_2^2+y_2^2) = \Ulp(d)$, and the same relation
holds for $\Ulp(y_2^2)$, thus
\[
|d-D| \leq 2\left[(1+c_{R,2})k_{R,2}+(1+c_{I,2})k_{I,2}\right]\Ulp(d);
\]
note that
\[
\left|\frac{A}{D}\right| \leq
\frac{c_{R,1}c_{R,2}+c_{I,1}c_{I,2}}{(c_{R,2}^-)^2+(c_{I,2}^-)^2}
\left|\frac{a}{d}\right|,
\]
thus we have
\[
\left|\frac{A}{D}\right|\frac{1}{d}|d-D| \leq
4\frac{c_{R,1}c_{R,2}+c_{I,1}c_{I,2}}{(c_{R,2}^-)^2+(c_{I,2}^-)^2}
\left((1+c_{R,2})k_{R,2}+(1+c_{I,2})k_{I,2}\right)\Ulp(x).
\]
Gathering the relevant inequalities, we show that the error on the real part
is bounded in the following way
\begin{equation*}
  \begin{split}
    \error(x) &\leq [c_R\\
    &\quad+\left((1+c_{R,1})k_{R,2}+(1+c_{R,2})k_{R,1}\right)2^{1+d}
    +\left((1+c_{I,1})k_{I,2}+(1+c_{I,2})k_{I,1}\right)2^{1+d'}\\
    &\quad+4\frac{c_{R,1}c_{R,2}+c_{I,1}c_{I,2}}{(c_{R,2}^-)^2+(c_{I,2}^-)^2}
    \left((1+c_{R,2})k_{R,2}+(1+c_{I,2})k_{I,2}\right)]
    \Ulp(x).
  \end{split}
\end{equation*}

An analog process gives
\begin{equation*}
  \begin{split}
    \error(y) &\leq [c_R\\
    &\quad+\left((1+c_{I,1})k_{R,2}+(1+c_{R,2})k_{I,1}\right)2^{1+\delta}
    +\left((1+c_{R,1})k_{I,2}+(1+c_{I,2})k_{R,1}\right)2^{1+\delta'}\\
    &\quad+4\frac{c_{R,1}c_{R,2}+c_{I,1}c_{I,2}}{(c_{R,2}^-)^2+(c_{I,2}^-)^2}
    \left((1+c_{R,2})k_{R,2}+(1+c_{I,2})k_{I,2}\right)]
    \Ulp(y),
  \end{split}
\end{equation*}
with $\delta=\Exp(y_1x_2)-\Exp(b)$ and $\delta'=\Exp(x_1y_2)-\Exp(b)$.

Note that, when $z_1$ and $z_2$ are the rounded values of $\widetilde{z_1}$
and $\widetilde{z_2}$ respectively, with rounding towards plus infinity, we
have the much simpler inequalities:
\begin{align*}
\error(x) &\leq c_R + 2^{3+d} + 2^{3+d'} + 2^6\\
\error(y) &\leq c_R + 2^{3+\delta} + 2^{3+\delta'} + 2^6.
\end{align*}

At last, remark that $x_1x_2$ and $y_1y_2$ cannot have the same sign if
$y_1x_2$ and $-x_1y_2$ do have the same sign, thus there is always a
cancellation sometimes in real part, sometimes in imaginary part of the
division.


\section {Algorithms}

This section describes in detail the algorithms used in \mpc, together with the error analysis that allows to prove that the results are correct in the {\mpc} semantics: The input numbers are assumed to be exact, and the output corresponds to the exact result rounded in the desired direction.


\subsection {\texttt {mpc\_sqrt}}

Let $z = x + i y$.

Let $w = \sqrt { \frac {|x| + \sqrt {x^2 + y^2}}{2}}$ and
$t = \frac {y}{2w}$. Then $(w + it)^2 = |x| + iy$, and with the branch cut on the negative real axis we obtain
\[
\sqrt z = \left\{
\begin {array}{cl}
w + i t & \text {if } x > 0 \\
t + i w & \text {if } x < 0, y > 0 \\
-t - i w & \text {if } x < 0, y < 0
\end {array}
\right.
\]

$w$ is rounded down. $\sqrt {x^2 + y^2}$ is computed with an error of \ulp{1}; $|x|$ is added with an error of \ulp{1}, since both terms are positive. The generic error of the real square root in the special case that the argument was rounded down is \ulp{1}, so that the total error in computing $w$ is \ulp{3}.

$t$ is rounded up. The generic error of real division, applied to an error of \ulp{3} for $w$ and \ulp{0} for $y$ implies an error of \ulp{7}.


\subsection {\texttt {mpc\_log}}

Let $z = x + i y$. Then $\log (z) = \frac {1}{2} \log (x^2 + y^2) + i \atantwo (y, x)$. The imaginary part is computed by a call to the corresponding {\mpfr} function.

Let $w = \log (x^2 + y^2)$, rounded down. The error of the complex norm is \ulp{1}. The generic error of the real logarithm is then given by \ulp{$2^{2 - e_w} + 1$}, where $e_w$ is the exponent of $w$. For $e_w \geq 2$, this is bounded by \ulp{2} or 2~digits; otherwise, it is bounded by \ulp{$2^{3 - e_w}$} or $3 - e_w$ digits.

\end {document}
